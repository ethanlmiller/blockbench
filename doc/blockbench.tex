\documentclass[letterpaper,11pt]{article}
\usepackage[text={6.5in,9in},centering]{geometry}
\usepackage{epsfig,times}
\usepackage{graphicx,xspace,url,verbatim}
\usepackage{tabularx}
\usepackage{algorithm}
\usepackage{algpseudocode}

\newcommand{\etal}{et~al.\xspace}
\newcommand{\eg}{e.g.\xspace}
\newcommand{\ie}{i.e.\xspace}

\newcommand{\blockbench}{\texttt{blockbench}\xspace}
\newcommand{\ioload}{\texttt{ioload}\xspace}
\newcommand{\iofill}{\texttt{iofill}\xspace}

\title{\blockbench: A Block-Based Load Generator and Benchmark Toolset}

\author{}

\begin{document}

\maketitle

\begin{abstract}
  \blockbench is a set of tools to generate workloads for benchmarking block-based
  storage systems.  The tool set supports a range of timing options for issuing I/Os,
  control over the read/write ratio, and the ability to set both the compression level and
  deduplication level of the written data, as well as other features.  It is the ability
  to finely control timing and data content of the written blocks that makes this
  toolset more relevant for modern flash-based storage systems than existing tools, which
  often assume that the block-based storage system won't perform compression or deduplication.
\end{abstract}

\section{Introduction}

\section{Building the Tools}

The toolset uses \verb+autoconf+ and \verb+libtool+ to build the binaries.  Building on Linux
should be straightforward: \\
\begin{verbatim}
./configure
make
\end{verbatim}

However, the tools require the \verb+sg3_utils+ include files, which may not be installed on
your system.  You can either install them as root, making them available to everyone, or install
them in your home directory.  If you do the latter, you'll probably need to add C preprocessor
flags so that the build system can find the include files: \\
\verb+./configure CPPFLAGS='/users/me/include'+ \\
You can get a copy of \verb+sg3_utils+ at \url{http://sg.danny.cz/sg/sg3_utils.html}, or you
can install it using standard installation tools on many recent Linux distributions.

\section{Using the Tools}

There are three steps in using \blockbench on a system:
\begin{enumerate}
\item Set up the environment.
\item Prefill data into the part of the device(s) on which benchmarks will run.
\item Run the actual performance tests.
\end{enumerate}

\subsection{Setting up the Test Environment}

\blockbench can be used in any environment that supports block-based I/O, as long as it
also supports ``direct'' I/O that bypasses the operating system's caches.  We bypass the
cache because doing so results in more accurate performance numbers: if I/O is done to the
cache rather than the device, you can get very good results until you run out of cache
space.

Since there are many different environments in which \blockbench can be used, it's impossible
to list all the ways in which the environment should be set up.  However, it's important that
there be one or more files corresponding to the raw devices on which \blockbench will be
used.

Typically, there's one raw device per file; however, it's certainly possible to have
multiple devices corresponding to a single device.  If this is done, care should be taken
to ensure that multiple runs of \ioload don't interfere with one another.

\subsection{Prefilling Data}
\label{sec:prefill}

The tools in \blockbench typically do random reads across the region being used in the
test.  To ensure that a random read returns ``good'' data, it's necessary to preload the
region with data.  Preloading data can also allow you to test the ability of the storage
device to compress and deduplicate data by writing data patterns that are amenable to data
reduction.

The \iofill tool can be used to prefill a region with data.  \iofill allows you to control
the type of data (compressibility) and the ``space'' of valid blocks, which determines the
level of deduplication that might be available.  It also allows you to control I/O size
and the number of threads that run simultaneously (each individual thread waits for an
outstanding I/O to complete before issuing the next one).

The data types supported by \iofill, and the other \blockbench tools are:

\begin{tabular}{|l|l|} \hline
\textbf{Abbrv}             & \textbf{Data type} \\ \hline \hline
\textbf{c}                 & compressible data \\ \hline
\textbf{h1} or \textbf{h}  & data compressible to approx 1\% of original size \\ \hline
\textbf{h10}               & data compressible to approx 10\% of original \\ \hline
\textbf{h20}               & data compressible to approx 20\% of original \\ \hline
\textbf{h40}               & data compressible to approx 40\% of original \\ \hline
\textbf{e}                 & email data from \texttt{.pst} files \\ \hline
\textbf{v}                 & virtual machine data (\emph{default}) \\ \hline
\textbf{u}                 & uncompressible data \\ \hline
\end{tabular}

There are three types of options, test options (preceded by \verb+-t+), module options
(preceded by \verb+-m+), and detail options (preceded by \verb+-d+) that control the
I/Os for which detailed information is reported.  The test options control the number,
size, and distribution of I/Os.  A sample set of test options would be: \\
\verb+-t size=16384,nrep=1,skip=17+ \\
This tells \iofill to use 16\,KB I/Os, fill the region only once (the default), and skip
17~I/Os between requests.  The \verb+skip+ parameter can be used to prevent \iofill from
writing data purely sequentially; some I/O systems are very good at prediction, and using
a high value for \verb+skip+ makes this less useful.  If the \verb+skip+ option is used,
be aware that the entire region may not get filled if the interval is chosen poorly.  For
example, if the region is 1\,GB and skip is 1\,MB, the system will only write the blocks at
1\,MB intervals.

The module options can be used to change the performance of \iofill and the region that it
uses; the data type is also specified here.  A sample line might be:
\\ \verb+-m threads=128,dtype=v,minoff=4,maxoff=4096,dedupblks=200+ \\ This would tell
\iofill to use 128~threads (maximum concurrency level of writes) to fill the region from
4\,KB to 4096\,KB.  It would use data drawn from a predefined virtual memory data set, and
would have a total of 200,000 unique blocks in the data set.  The \verb+dedupblks+
parameter controls the amount of deduplication; lower numbers for the size of the dedup
set will result in more deduplication.

An overall command line for \iofill might then be: \\
\verb+iofill -t size=16384 -m threads=128,dtype=v,minoff=4,maxoff=4096 /dev/vol1 /dev/vol2+ \\
This would operate on both \verb+/dev/vol1+ and \verb+/dev/vol2+.

\subsection{Performance Testing}

Now that the testing regions have been filled with data, we can start the performance
testing.  To do this, we'll use \ioload, a tool that shares a lot of library code with
\iofill, but supports a lot more I/O patterns.

A sample run, with 75\% read and 25\% write, might use this command line: \\
\begin{verbatim}
ioload -n -o ioload-16k-rd75.out \
       -i pct=.75,size=16384,op=read \
       -i pct=.25,size=16384,op=write,dtype=v \
       -m initthreads=16,maxthreads=32,incrthreads=2,interval=10,maxlat=4000 \
       /dev/vol1 /dev/vol2
\end{verbatim}

This command line would ignore I/O errors (\verb+-n+), write output results to
\verb+ioload-16k-rd75.out+, and operate on \verb+/dev/vol1+ and \verb+/dev/vol2+.  Each of
the \verb+-i+ lines is an I/O specifier.  The first line specifies that 75\% of the I/Os
will be 16\,KB reads, and the second says that 25\% will be 16\,KB writes, and that each
write will generate data using the predefined VM data (see Section~\ref{sec:prefill}).
The module-specific options say that the test will start with 16~threads, building by
2~threads every 10 seconds, to a total of 32~threads.  The test will stop when the 90th
percentile latency exceeds 4\,ms (4000\,$\mu$s).  The system will run for 60~seconds
before starting to report results, though \ioload generates load during the warmup period,
and external tools can measure performance during this time.

\end{document}
